%% LyX 2.3.6 created this file.  For more info, see http://www.lyx.org/.
%% Do not edit unless you really know what you are doing.
\documentclass[ruled]{article}
\usepackage{courier}
\usepackage[T1]{fontenc}
\usepackage[latin9]{inputenc}
\usepackage[letterpaper]{geometry}
\geometry{verbose}
\setcounter{secnumdepth}{5}
\usepackage{color}
\usepackage{enumitem}
\usepackage{algorithm2e}
\usepackage{amsmath}
\usepackage{amssymb}
\usepackage[unicode=true,
 bookmarks=false,
 breaklinks=false,pdfborder={0 0 1},backref=section,colorlinks=true]
 {hyperref}

\makeatletter

%%%%%%%%%%%%%%%%%%%%%%%%%%%%%% LyX specific LaTeX commands.
\providecommand{\LyX}{\texorpdfstring%
  {L\kern-.1667em\lower.25em\hbox{Y}\kern-.125emX\@}
  {LyX}}

%%%%%%%%%%%%%%%%%%%%%%%%%%%%%% Textclass specific LaTeX commands.
\newlength{\lyxlabelwidth}      % auxiliary length 

\@ifundefined{date}{}{\date{}}
%%%%%%%%%%%%%%%%%%%%%%%%%%%%%% User specified LaTeX commands.
\definecolor{mygreen}{rgb}{0,0.6,0}
\definecolor{mygray}{rgb}{0.5,0.5,0.5}
\definecolor{mymauve}{rgb}{0.58,0,0.82}

\makeatother

\usepackage{listings}
\lstset{backgroundcolor={\color{white}},
basicstyle={\footnotesize\ttfamily},
breakatwhitespace=false,
breaklines=true,
captionpos=b,
commentstyle={\color{mygreen}},
deletekeywords={...},
escapeinside={\%*}{*)},
extendedchars=true,
frame=shadowbox,
keepspaces=true,
keywordstyle={\color{blue}},
language=Python,
morekeywords={*,...},
numbers=none,
numbersep=5pt,
numberstyle={\tiny\color{mygray}},
rulecolor={\color{black}},
showspaces=false,
showstringspaces=false,
showtabs=false,
stepnumber=1,
stringstyle={\color{mymauve}},
tabsize=2}
\begin{document}
\global\long\def\reals{\mathbf{\mathbb{R}}}%
\global\long\def\integers{\mathbf{Z}}%
\global\long\def\naturals{\mathbf{N}}%
\global\long\def\rationals{\mathbf{Q}}%
\global\long\def\ca{\mathcal{A}}%
\global\long\def\cb{\mathcal{B}}%
\global\long\def\cc{\mathcal{C}}%
\global\long\def\cd{\mathcal{D}}%
\global\long\def\ce{\mathcal{E}}%
\global\long\def\cf{\mathcal{F}}%
\global\long\def\cg{\mathcal{G}}%
\global\long\def\ch{\mathcal{H}}%
\global\long\def\ci{\mathcal{I}}%
\global\long\def\cj{\mathcal{J}}%
\global\long\def\ck{\mathcal{K}}%
\global\long\def\cl{\mathcal{L}}%
\global\long\def\cm{\mathcal{M}}%
\global\long\def\cn{\mathcal{N}}%
\global\long\def\co{\mathcal{O}}%
\global\long\def\cp{\mathcal{P}}%
\global\long\def\cq{\mathcal{Q}}%
\global\long\def\calr{\mathcal{R}}%
\global\long\def\cs{\mathcal{S}}%
\global\long\def\ct{\mathcal{T}}%
\global\long\def\cu{\mathcal{U}}%
\global\long\def\cv{\mathcal{V}}%
\global\long\def\cw{\mathcal{W}}%
\global\long\def\cx{\mathcal{X}}%
\global\long\def\cy{\mathcal{Y}}%
\global\long\def\cz{\mathcal{Z}}%
\global\long\def\pr{\mathbb{P}}%
\global\long\def\predsp{\cy}%
\global\long\def\outsp{\cy}%
\global\long\def\prxy{P_{\cx\times\cy}}%
\global\long\def\prx{P_{\cx}}%
\global\long\def\prygivenx{P_{\cy\mid\cx}}%
\global\long\def\ex{\mathbb{E}}%
\global\long\def\var{\textrm{Var}}%
\global\long\def\cov{\textrm{Cov}}%
\global\long\def\sgn{\textrm{sgn}}%
\global\long\def\sign{\textrm{sign}}%
\global\long\def\kl{\textrm{KL}}%
\global\long\def\law{\mathcal{L}}%
\global\long\def\eps{\varepsilon}%
\global\long\def\as{\textrm{ a.s.}}%
\global\long\def\io{\textrm{ i.o.}}%
\global\long\def\ev{\textrm{ ev.}}%
\global\long\def\convd{\stackrel{d}{\to}}%
\global\long\def\convp{\stackrel{P}{\to}}%
\global\long\def\eqd{\stackrel{d}{=}}%
\global\long\def\del{\nabla}%
\global\long\def\loss{V}%
\global\long\def\risk{R}%
\global\long\def\etal{\textrm{et. al.}}%
\global\long\def\tr{\operatorname{tr}}%
\global\long\def\trace{\operatorname{trace}}%
\global\long\def\diag{\text{diag}}%
\global\long\def\rank{\text{rank}}%
\global\long\def\linspan{\text{span}}%
\global\long\def\proj{\text{Proj}}%
\global\long\def\argmax{\operatornamewithlimits{arg\, max}}%
\global\long\def\argmin{\operatornamewithlimits{arg\, min}}%
\global\long\def\bfx{\mathbf{x}}%
\global\long\def\bfy{\mathbf{y}}%
\global\long\def\bfl{\mathbf{\lambda}}%
\global\long\def\bfm{\mathbf{\mu}}%
\global\long\def\calL{\mathcal{L}}%
\global\long\def\vw{\boldsymbol{w}}%
\global\long\def\vx{\boldsymbol{x}}%
\global\long\def\vxi{\boldsymbol{\xi}}%
\global\long\def\valpha{\boldsymbol{\alpha}}%
\global\long\def\vbeta{\boldsymbol{\beta}}%
\global\long\def\vsigma{\boldsymbol{\sigma}}%
\global\long\def\vmu{\boldsymbol{\mu}}%
\global\long\def\vtheta{\boldsymbol{\theta}}%
\global\long\def\vd{\boldsymbol{d}}%
\global\long\def\vs{\boldsymbol{s}}%
\global\long\def\vt{\boldsymbol{t}}%
\global\long\def\vh{\boldsymbol{h}}%
\global\long\def\ve{\boldsymbol{e}}%
\global\long\def\vf{\boldsymbol{f}}%
\global\long\def\vg{\boldsymbol{g}}%
\global\long\def\vr{\boldsymbol{r}}%
\global\long\def\vz{\boldsymbol{z}}%
\global\long\def\vk{\boldsymbol{k}}%
\global\long\def\va{\boldsymbol{a}}%
\global\long\def\vb{\boldsymbol{b}}%
\global\long\def\vv{\boldsymbol{v}}%
\global\long\def\vy{\boldsymbol{y}}%
\global\long\def\hil{\ch}%
\global\long\def\rkhs{\hil}%
\global\long\def\median{\text{Median}}%
\global\long\def\ipw{\text{ipw}}%
\global\long\def\ipwsn{\ensuremath{\text{ipwsn}}}%
\global\long\def\myfoot#1{\let\thefootnote\relax\footnotetext{\tiny{#1}}}%
% Causal inference
\global\long\def\ate{\text{ATE}}%
\global\long\def\unif{\text{Unif}}%
\global\long\def\bern{\text{Bern}}%
\global\long\def\indep{\perp\!\!\! \perp}%
\global\long\def\ind#1{\mathds{1}\left[#1\right]}%
\global\long\def\se#1{\text{SE}\left(#1\right)}%
\global\long\def\corr{\text{Corr}}%
\global\long\def\expit{\text{expit}}%
\global\long\def\snipw{\ensuremath{\text{sn\_ipw}}}%
\global\long\def\bias{\ensuremath{\text{Bias}}}%
\global\long\def\sd{\textrm{SD}}%
\global\long\def\opt{\text{opt}}%
\global\long\def\dm{\ensuremath{\text{dm}}}%
\global\long\def\poem{\text{POEM}}%


\title{Tools and Techniques in Machine Learning\\
Homework 1: Missing data and inverse propensity weighting}

\maketitle
\textbf{Instructions}: Your answers to the questions below, including
plots and mathematical work, should be submitted as a single PDF file.
It's preferred that you write your answers using software that typesets
mathematics (e.g. \LaTeX , \LyX , or Jupyter), though if you need
to you may scan handwritten work. For submission, you can also export
your Jupyter notebook and merge that PDF with your PDF for the written
solutions into one file. \textbf{Don't forget to complete the Jupyter
notebook as well, for the programming part of this assignment}.

\section{Estimators for missing at random (MAR)}

All questions below pertain to the missing at random (MAR) setting.
Let's review the MAR setup: $\left(X,R,Y\right),\left(X_{1},R_{1},Y_{1}\right),\ldots,(X_{n},R_{n},Y_{n})$
are i.i.d. with covariate $X\in\cx$, response indicator $R\in\left\{ 0,1\right\} $,
and response $Y\in\reals$. Under MAR, we assume that $R\indep Y\mid X$,
and the response probability is given by $\pr\left(R=1\mid X=x\right)=\pi(x)\in(0,1]$,
where $\pi(x)$ is the propensity score function. The $Y_{i}$'s corresponding
to $R_{i}=0$ are unobserved. The missing data problem is to estimate
$\ex Y$ without using the unobserved $Y_{i}$'s, which is equivalent
to using only $\left(X_{1},R_{1},R_{1}Y_{1}\right),\ldots,\left(X_{n},R_{n},R_{n}Y_{n}\right)$. 

\subsection{Total inverse propensity weight for observations has expectation
$n$}

Let $W_{i}=\frac{1}{\pi(X_{i})}$ be the inverse propensity weight
for $Y_{i}$.
\begin{enumerate}
\item Show that 
\[
\ex\left[\sum_{i=1}^{n}W_{i}R_{i}\right]=n.
\]

\end{enumerate}

\subsection{Complete case estimator is not consistent for MAR setting.}
\begin{enumerate}
\item The complete case mean is defined as $\hat{\mu}_{\text{cc}}=\sum_{i=1}^{n}R_{i}Y_{i}/\sum_{i=1}^{n}R_{i}$.
Show that under the MAR assumption,
\[
\hat{\mu}_{\text{cc}}\convp\frac{\ex\left[\pi(X)\mu(X)\right]}{\ex\left[\pi(X)\right]},
\]
where $\mu(x)=\ex\left[Y|X=x\right]$. Assume\footnote{The first first inequality is true if $Y$ is bounded, which is reasonable
in all our applications, and the second inequality is clearly true
since $R\in\left\{ 0,1\right\} $.} that $\ex\left|RY\right|<\infty$ and $\ex\left|R\right|<\infty$.
{[}Hint: The weak law of large numbers (WLLN) states that if $Y,Y_{1},\ldots,Y_{n}$
are i.i.d. with $\ex\left|Y\right|<\infty$, then $\frac{1}{n}\sum_{i=1}^{n}Y_{i}\convp\ex Y$.
Apply the WLLN on the numerator and denominator separately, and then
apply Slutsky's Theorem, which states that if $X_{n}\convp a$ and
$Y_{n}\convp b$ for constants $a$ and $b\neq0$, then $\frac{X_{n}}{Y_{n}}\convp\frac{a}{b}.${]}

\end{enumerate}

\begin{enumerate}[resume]
\item Recall the SeaVan1 distribution from lecture (which is based on Example
1 in \cite{seaman-2018-introd-to}):
\begin{eqnarray*}
X & \sim & \unif\left(\left\{ 0,1,2\right\} \right)\\
Y\mid X=x & \sim & \cn(x,1)\\
R\mid X=x & \sim & \expit(4-4x),
\end{eqnarray*}
where $\expit(x)=1/\left(1+e^{-x}\right)$. What does the complete
case mean converge to for the SeaVan1 distribution {[}with at least
2 decimal places accuracy{]}? What is $\ex Y$?. The large gap between
the two is why we need to develop more sophisticated estimators to
handle response bias.

\end{enumerate}

\subsection{IPW estimator is not equivariant}

Suppose $\cd$ represents the dataset $(X_{1},R_{1},R_{1}Y_{1}),\ldots,(X_{n},R_{n},R_{n}Y_{n})$.
For any $a\in\reals$, we'll write $\cd-a$ for the dataset $(X_{1},R_{1},R_{1}\left(Y_{1}-a\right)),\ldots,(X_{n},R_{n},R_{n}\left(Y_{n}-a\right))$,
which is the same as $\cd$, but with each $Y$ value shifted by $a$.
We say that an estimator $\hat{\mu}(\cd)$ is \textbf{equivariant}
if $\hat{\mu}(\cd-a)=\hat{\mu}(\cd)-a$ for any $\cd$. In other words,
subtracting $a$ from all responses $Y_{i}$ just shifts the estimate
by the same amount $a$. (This definition is based on \cite[Ch 3]{lehmann-1998-theory-point-est}.)
\begin{enumerate}
\item Show that the self-normalized IPW estimator $\hat{\mu}_{\snipw}$
is equivariant. Explain why this implies the complete case estimator
$\hat{\mu}_{\text{cc}}$ is also equivariant.

\end{enumerate}
\begin{enumerate}[resume]
\item Show that $\hat{\mu}_{\ipw}(\cd-a)=\hat{\mu}_{\ipw}(\cd)-\frac{a}{n}\sum_{i=1}^{n}\frac{R_{i}}{\pi(X_{i})}$
and demonstrate that $\hat{\mu}_{\ipw}$ is generally not equivariant
(though it is if $\pi(x)\equiv1$). 

\item Consider the estimator $\hat{\mu}_{\ipw,a}(\cd):=\hat{\mu}_{\ipw}(\cd-a)+a$.
Show that $\hat{\mu}_{\ipw,a}(\cd)$ is an unbiased estimator of $\ex Y$.
{[}Hint: We already know that $\hat{\mu}_{\ipw}(\cd)$ is an unbiased
estimator of $\ex Y$.{]} \\
\textbf{Remark:} By varying $a$, we can get a whole collection of
unbiased estimators of $\ex Y$. Some will be better than others.
We'll revisit this setup in our next homework, where we'll view $\hat{\mu}_{\ipw,a}(\cd)$
as a control variate adjustment of $\hat{\mu}_{\text{\ensuremath{\ipw}}}$,
with the hope that a judicious choice of $a$ will lead to an estimator
with reduced variance.

\end{enumerate}

\bibliographystyle{amsalpha}
\bibliography{refs}

\end{document}
