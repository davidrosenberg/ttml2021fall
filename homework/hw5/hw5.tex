%% LyX 2.3.6 created this file.  For more info, see http://www.lyx.org/.
%% Do not edit unless you really know what you are doing.
\documentclass[ruled]{article}
\usepackage{courier}
\usepackage[T1]{fontenc}
\usepackage[latin9]{inputenc}
\usepackage[letterpaper]{geometry}
\geometry{verbose}
\setcounter{secnumdepth}{5}
\usepackage{color}
\usepackage{enumitem}
\usepackage{algorithm2e}
\usepackage{amsmath}
\usepackage{amssymb}
\usepackage[unicode=true,
 bookmarks=false,
 breaklinks=false,pdfborder={0 0 1},backref=section,colorlinks=true]
 {hyperref}

\makeatletter

%%%%%%%%%%%%%%%%%%%%%%%%%%%%%% LyX specific LaTeX commands.
\providecommand{\LyX}{\texorpdfstring%
  {L\kern-.1667em\lower.25em\hbox{Y}\kern-.125emX\@}
  {LyX}}

%%%%%%%%%%%%%%%%%%%%%%%%%%%%%% Textclass specific LaTeX commands.
\newlength{\lyxlabelwidth}      % auxiliary length 

\@ifundefined{date}{}{\date{}}
%%%%%%%%%%%%%%%%%%%%%%%%%%%%%% User specified LaTeX commands.
\definecolor{mygreen}{rgb}{0,0.6,0}
\definecolor{mygray}{rgb}{0.5,0.5,0.5}
\definecolor{mymauve}{rgb}{0.58,0,0.82}

\makeatother

\usepackage{listings}
\lstset{backgroundcolor={\color{white}},
basicstyle={\footnotesize\ttfamily},
breakatwhitespace=false,
breaklines=true,
captionpos=b,
commentstyle={\color{mygreen}},
deletekeywords={...},
escapeinside={\%*}{*)},
extendedchars=true,
frame=shadowbox,
keepspaces=true,
keywordstyle={\color{blue}},
language=Python,
morekeywords={*,...},
numbers=none,
numbersep=5pt,
numberstyle={\tiny\color{mygray}},
rulecolor={\color{black}},
showspaces=false,
showstringspaces=false,
showtabs=false,
stepnumber=1,
stringstyle={\color{mymauve}},
tabsize=2}
\begin{document}
\global\long\def\reals{\mathbf{\mathbb{R}}}%
\global\long\def\integers{\mathbf{Z}}%
\global\long\def\naturals{\mathbf{N}}%
\global\long\def\rationals{\mathbf{Q}}%
\global\long\def\ca{\mathcal{A}}%
\global\long\def\cb{\mathcal{B}}%
\global\long\def\cc{\mathcal{C}}%
\global\long\def\cd{\mathcal{D}}%
\global\long\def\ce{\mathcal{E}}%
\global\long\def\cf{\mathcal{F}}%
\global\long\def\cg{\mathcal{G}}%
\global\long\def\ch{\mathcal{H}}%
\global\long\def\ci{\mathcal{I}}%
\global\long\def\cj{\mathcal{J}}%
\global\long\def\ck{\mathcal{K}}%
\global\long\def\cl{\mathcal{L}}%
\global\long\def\cm{\mathcal{M}}%
\global\long\def\cn{\mathcal{N}}%
\global\long\def\co{\mathcal{O}}%
\global\long\def\cp{\mathcal{P}}%
\global\long\def\cq{\mathcal{Q}}%
\global\long\def\calr{\mathcal{R}}%
\global\long\def\cs{\mathcal{S}}%
\global\long\def\ct{\mathcal{T}}%
\global\long\def\cu{\mathcal{U}}%
\global\long\def\cv{\mathcal{V}}%
\global\long\def\cw{\mathcal{W}}%
\global\long\def\cx{\mathcal{X}}%
\global\long\def\cy{\mathcal{Y}}%
\global\long\def\cz{\mathcal{Z}}%
\global\long\def\pr{\mathbb{P}}%
\global\long\def\predsp{\cy}%
\global\long\def\outsp{\cy}%
\global\long\def\prxy{P_{\cx\times\cy}}%
\global\long\def\prx{P_{\cx}}%
\global\long\def\prygivenx{P_{\cy\mid\cx}}%
\global\long\def\ex{\mathbb{E}}%
\global\long\def\var{\textrm{Var}}%
\global\long\def\cov{\textrm{Cov}}%
\global\long\def\sgn{\textrm{sgn}}%
\global\long\def\sign{\textrm{sign}}%
\global\long\def\kl{\textrm{KL}}%
\global\long\def\law{\mathcal{L}}%
\global\long\def\eps{\varepsilon}%
\global\long\def\as{\textrm{ a.s.}}%
\global\long\def\io{\textrm{ i.o.}}%
\global\long\def\ev{\textrm{ ev.}}%
\global\long\def\convd{\stackrel{d}{\to}}%
\global\long\def\convp{\stackrel{P}{\to}}%
\global\long\def\eqd{\stackrel{d}{=}}%
\global\long\def\del{\nabla}%
\global\long\def\loss{V}%
\global\long\def\risk{R}%
\global\long\def\etal{\textrm{et. al.}}%
\global\long\def\tr{\operatorname{tr}}%
\global\long\def\trace{\operatorname{trace}}%
\global\long\def\diag{\text{diag}}%
\global\long\def\rank{\text{rank}}%
\global\long\def\linspan{\text{span}}%
\global\long\def\proj{\text{Proj}}%
\global\long\def\argmax{\operatornamewithlimits{arg\, max}}%
\global\long\def\argmin{\operatornamewithlimits{arg\, min}}%
\global\long\def\bfx{\mathbf{x}}%
\global\long\def\bfy{\mathbf{y}}%
\global\long\def\bfl{\mathbf{\lambda}}%
\global\long\def\bfm{\mathbf{\mu}}%
\global\long\def\calL{\mathcal{L}}%
\global\long\def\vw{\boldsymbol{w}}%
\global\long\def\vx{\boldsymbol{x}}%
\global\long\def\vxi{\boldsymbol{\xi}}%
\global\long\def\valpha{\boldsymbol{\alpha}}%
\global\long\def\vbeta{\boldsymbol{\beta}}%
\global\long\def\vsigma{\boldsymbol{\sigma}}%
\global\long\def\vmu{\boldsymbol{\mu}}%
\global\long\def\vtheta{\boldsymbol{\theta}}%
\global\long\def\vd{\boldsymbol{d}}%
\global\long\def\vs{\boldsymbol{s}}%
\global\long\def\vt{\boldsymbol{t}}%
\global\long\def\vh{\boldsymbol{h}}%
\global\long\def\ve{\boldsymbol{e}}%
\global\long\def\vf{\boldsymbol{f}}%
\global\long\def\vg{\boldsymbol{g}}%
\global\long\def\vr{\boldsymbol{r}}%
\global\long\def\vz{\boldsymbol{z}}%
\global\long\def\vk{\boldsymbol{k}}%
\global\long\def\va{\boldsymbol{a}}%
\global\long\def\vb{\boldsymbol{b}}%
\global\long\def\vv{\boldsymbol{v}}%
\global\long\def\vy{\boldsymbol{y}}%
\global\long\def\hil{\ch}%
\global\long\def\rkhs{\hil}%
\global\long\def\median{\text{Median}}%
\global\long\def\ipw{\text{ipw}}%
\global\long\def\ipwsn{\ensuremath{\text{ipwsn}}}%
\global\long\def\myfoot#1{\let\thefootnote\relax\footnotetext{\tiny{#1}}}%
% Causal inference
\global\long\def\ate{\text{ATE}}%
\global\long\def\unif{\text{Unif}}%
\global\long\def\bern{\text{Bern}}%
\global\long\def\indep{\perp\!\!\! \perp}%
\global\long\def\ind#1{\mathds{1}\left[#1\right]}%
\global\long\def\se#1{\text{SE}\left(#1\right)}%
\global\long\def\corr{\text{Corr}}%
\global\long\def\expit{\text{expit}}%
\global\long\def\snipw{\ensuremath{\text{sn\_ipw}}}%
\global\long\def\bias{\ensuremath{\text{Bias}}}%
\global\long\def\sd{\textrm{SD}}%
\global\long\def\opt{\text{opt}}%
\global\long\def\dm{\ensuremath{\text{dm}}}%
\global\long\def\poem{\text{POEM}}%


\title{Tools and Techniques for Machine Learning\\
Homework 5}

\maketitle
\textbf{Instructions}: Your answers to the questions below, including
plots and mathematical work, should be submitted as a single PDF file.
It's preferred that you write your answers using software that typesets
mathematics (e.g. \LaTeX , \LyX , or Jupyter), though if you need
to you may scan handwritten work. For submission, you can also export
your Jupyter notebook and merge that PDF with your PDF for the written
solutions into one file. \textbf{Don't forget to complete the Jupyter
notebook as well, for the programming part of this assignment}. 


\section{Calibration error}

We the defined \textbf{calibration error} of $f:\cx\to[0,1]$ is as
\[
\text{CE}(f)=\left(\ex\left[\left(f(X)-\pr[Y=1\mid f(X)]\right)^{2}\right]\right)^{1/2}
\]
and we defined the \textbf{integrated squared error as}
\[
\text{ISE}(f)=\left(\ex\left[\left(f(X)-\pr[Y=1\mid X]\right)^{2}\right]\right)^{1/2}.
\]
We claimed that without any knowledge or assumption about $\pr[Y=1\mid X]$,
such as it being in some smooth class of functions, it can be impossible
to get a good estimate of ISE$(f)$. The essence of the issue is that
we need to have data from all possible values of $X$ to estimate
$\pr\left[Y=1\mid X\right]$, and this is a problem if $X$ is a continuous
variable. In fact, we'll have the same problem estimating $\text{CE}(f)$
if $f(X)$ take continuous values (or takes uncountably many different
values). 
\begin{enumerate}
\item Show that if $f:\cx\to\left[0,1\right]$ is an injective function
(i.e. $x\neq x'\implies f(x)\neq f(x')$ for any $x,x'\in\cx$), then
$\text{CE}(f)=\text{ISE}(f)$. {[}Hint: Let $g(x)=\pr\left[Y=1\mid f(X)=f(x)\right]$and
let $h(x):=\pr\left[Y=1\mid X=x\right]$ and show that $g(x)=h(x)$
for all $x\in\cx$.{]} {[}Discussion: When $f$ is injective, we can
say that $f(x)$ maintains all the information in $x$. The implication
of this question is that if we want to be able to estimate $\text{CE}(f)$,
we're going to need to make some assumptions about $f$. The assumption
that is typically made is that $f(x)$ takes on only finitely many
different values. If this isn't the case, we can approximate $f$
by ``binning'', as discussed in lecture, then estimate the CE for
the binned $f$.{]}

\end{enumerate}


\section{Plug-in estimator of calibration error }

Suppose we want to estimate the calibration error of $f:\cx\to[0,1]$,
but $f(x)$ takes too many different values to estimate $\pr\left[Y=1\mid f(X)=f(x)\right]$
for each. We decide to approximate $f$ with a ``binned version''
$f_{\cb}(x)$ that takes only finitely many values, as follows: Let
$\cb$ be a partition of $[0,1]$ into disjoint sets (i.e. ``bins'')
$I_{1},\dots,I_{B}$, and define
\[
f_{\cb}(x)=\ex[f(X)\mid f(X)\in I_{b}]\qquad\text{where }f(x)\in I_{b}.
\]
Without knowledge of the marginal distribution of $X$, we can't compute
$f_{\cb}(x)$. However, we'll assume we have a labeled sample $(X_{1},Y_{1}),\ldots,(X_{n},Y_{n})$,
and we'll use the natural estimate 
\[
\hat{f}_{\cb}(x):=\text{mean}\left\{ f(X_{i})\mid f(X_{i})\text{ and }f(x)\text{ are in the same bin}\right\} .
\]
We'll now define the ``plug-in estimator'' for the {[}squared{]}
CE of $f_{\cb}$ as
\[
\widehat{\text{CE}^{2}}(f_{\cb}):=\sum_{b=1}^{B}\hat{p}_{b}(\hat{f}_{\cb}(x_{b})-\hat{\mu}_{b})^{2},
\]
where $x_{b}$ is any value for which $f(x_{b})\in I_{b}$, $\hat{\mu}_{b}=\text{mean}\left\{ Y_{i}\mid f(X_{i})\in I_{b}\right\} $
and $\hat{p}_{b}=n_{b}/n$, where $n_{b}$ is the number of $X_{i}$'s
for which $f(X_{i})\in I_{b}$.
\begin{enumerate}
\item Assuming the partition $\cb$ is determined independently of the sample,
give an expression for $\text{CE}_{\infty}^{2}(f_{\cb})$, the limit
of $\widehat{\text{CE}}(f_{\cb})$ as $n\to\infty$. {[}Hint: You'll
want to define $\phi_{b}:=\pr\left(f(X)\in I_{b}\right)$, the probability
that a prediction is in bin $b$. You can also use the expression
$\ex\left[Y\mid f(X)\in I_{b}\right]$ in your answer.{]} (Just provide
the expression -- you don't have to prove it. But if you feel like
being rigorous, the proof is a straightforward application of the
weak law of large numbers, Slutsky's theorem, and the continuous mapping
theorem.)

\item Show that $\text{CE}_{\infty}^{2}(f_{\cb})=\left[\text{CE}(f_{\cb})\right]^{2}$.
Combined with our previous problem, this will imply that $\widehat{\text{CE}}(f_{\cb})$
is a consistent estimator of the calibration error of $f_{\cb}$.

\item {[}Optional -- no credit{]}Above, we assumed that the binning was
determined independently of our sample. Now suppose we use the same
sample to determine the bins $\cb$ as we use to compute $\widehat{\text{CE}}(f_{\cb})$.
Show that if the $f(X_{i})$ are distinct for all $i=1,\ldots,n$
and if we use quantile binning with $n$ bins, then $\widehat{\text{CE}}(f_{\cb})$
becomes the empirical Brier score of $f$ (i.e. mean squared error).
{[}This is bad if our goal is to estimate $\text{CE}(f)$, because
the Brier score is quite a different thing from calibration error,
as we've discussed.{]}

\end{enumerate}


\end{document}
